\documentclass{../extra/memoriaicci}
\titulo{TITULO DE MEMORIA}
\grado{ACTIVIDAD DE TITULACION PARA OPTAR AL TITULO DE INGENIERO CIVIL EN COMPUTACION E INFORMATICA}
\patrocinante{Nombre profesor}
\alumno{Nombre alumno}

\begin{document}

\maketitle

\begin{dedicatory}
dedicatoria \ldots
\end{dedicatory}

\begin{acknowledgment}

agradecimientos \ldots

\end{acknowledgment}

\tableofcontents
\listoffigures
\listoftables
%\lstlistoflistings

\begin{resumen}

resumen \ldots

\end{resumen}

% Los capítulos se incluyen como archivos .tex por separados
% En caso de no quere compilar alguno de los archivos, utilizar la funcion:

% \includeonly{filename1.tex,filename2.tex,...}

% Esto es útil para trabajar/compilar mas rapido y con un documento reducido
%!TEX root = indice.tex
\chapter{INTRODUCCION}

\section{Generalidades}

Generalidades \ldots

\section{Objetivos}

\subsection{Objetivo general}

\ldots

\subsection{Objetivos específicos}

\begin{itemize}
  \item Item 1
  \item Item 2
\end{itemize}

\subsection{Limitaciones y supuestos}

\ldots
%!TEX root = indice.tex
\chapter{ANTECEDENTES}

\section{Seccion}
\label{chap2:rest}

\subsection{Sub Sección}

Texto normal ``texto entre comillas''. Texto citado en refs.bib \cite{rthomas}.

\textbf{texto en negrita} texto con nota a pie de página\footnote{nota de pié de página}.

\begin{figure}[htb]
\centering
\includegraphics[scale=0.7]{../img/chap2_rest_diagram.pdf}
\caption{ejemplo inserciónd e imágen}
\label{fig:restDiagram}
\end{figure}

Ejem`lo de referencia de imagen' \ref{fig:restDiagram}

\subsubsection{Sub sub sección}

Ejemplo de listado de items

\begin{itemize}
\item item 1
\item item 2
\item item N
\end{itemize}

ejemplo de URL formateada

\url{http://www.biblioteca.cl/libros}

Ejemplo de listado

\lstset{language=XML}

\begin{lstlisting}[caption=Listado de libros obtenido del Servicio Web., label=xmllisting]
<?xml version="1.0"?>
<p:Libros xmlns:p="http://www.biblioteca.cl"
     xmlns:xlink="http://www.w3.org/1999/xlink">
  <Libro id="231" xlink:href="http://www.biblioteca.cl/libros/231"/>
  <Libro id="232" xlink:href="http://www.biblioteca.cl/libros/232"/>
  <Libro id="233" xlink:href="http://www.biblioteca.cl/libros/233"/>
  <Libro id="234" xlink:href="http://www.biblioteca.cl/libros/234"/>
</p:Libros>
\end{lstlisting}

Ejemplo de resaltado en listado:

\begin{lstlisting}[escapechar=!,caption=Libro en particular obtenido del Servicio Web.,label=xmlresourceforone]
<?xml version="1.0"?>
<p:Libro xmlns:p="http://www.biblioteca.cl"
    xmlns:xlink="http://www.w3.org/1999/xlink">
  <ID>234</ID>
  <Titulo>Algoritmos Avanzados</Titulo>
  <Resumen>Desarrollo avanzado de algoritmos</Resumen>
  !\hl{\mbox{\bf $<$Especificacion xlink:href="http://www.biblioteca.cl/libros/234/especificacion" /$>$ }}!
  <Valor moneda="CLP">2100</Valor>
  <Disponible>True</Disponible>
</p:Libro>
\end{lstlisting}




\bibliography{../extra/refs}

\end{document}

